% !TeX program = xelatex
\documentclass[11pt, a4paper]{article}

% --- UNIVERSAL PREAMBLE ---
\usepackage[utf8]{inputenc}
\usepackage[T1]{fontenc}
\usepackage{lmodern}
\usepackage[english]{babel}

\usepackage[a4paper, top=2.5cm, bottom=2.5cm, left=2.5cm, right=2.5cm]{geometry}
\usepackage{parskip} % Spacing between paragraphs
\usepackage{xcolor}
\usepackage{graphicx}
\usepackage{titlesec}
\usepackage{enumitem}

% Title Customization
\titleformat{\section}{\Large\bfseries\scshape\color{darkgray}}{}{0em}{}[\titlerule]
\titleformat{\subsection}{\large\bfseries\color{gray}}{}{0em}{}

\begin{document}

% --- PAGE TITLE ---
\begin{center}
    {\huge \textbf{My Journey}} \\
    \vspace{0.2cm}
    \textit{From the Rigor of Preparatory Classes to AI Research}
\end{center}

\vspace{0.8cm}

% --- INTRODUCTORY PARAGRAPH ---
\begin{center}
    \begin{minipage}{0.9\textwidth}
        \textit{My academic and professional path is defined by a constant search for balance: between theoretical abstraction and concrete action, between individual excellence and collective leadership. This journey has taken me from the blackboards of Paris to the jungles of Martinique, and finally to the cutting edge of Artificial Intelligence research.}
    \end{minipage}
\end{center}

\vspace{0.8cm}

% --- SECTION 1: CPGE ---
\section{2019-2021: The School of Resilience (Lycée Saint-Louis)}

After graduating from high school with the highest honors (\textbf{"Félicitations du Jury"}), my passion for the fundamental laws of physics and mathematics led me to the \textit{Classes Préparatoires}. I joined the prestigious \textbf{Lycée Saint-Louis} in Paris, initially in the PCSI track, before advancing to the \textbf{PSI*} (Star Class), joining a cohort of the nation's most promising students.

These two years were an intellectual forge. Beyond the complex equations of fluid mechanics or electromagnetism, I learned a vital lesson in **resilience**. The intensity of the workload taught me how to structure my thinking under pressure and to find comfort in the uncomfortable. It was here that I discovered that rigor is not just a method, but a mindset.

% Placeholder for PREPA PHOTO
\begin{figure}[h]
    \centering
    \framebox{\parbox{0.6\textwidth}{\centering \vspace{3cm} \textit{[Photo: Lycée Saint-Louis / Classroom atmosphere]} \vspace{3cm}}}
\end{figure}

This dedication culminated in the 2021 competitive entrance exams, where I achieved rankings that validated years of effort:
\begin{itemize}[label=\textbullet]
    \item \textbf{ENS Paris-Saclay}: 8\textsuperscript{th}
    \item \textbf{Mines-Ponts}: 33\textsuperscript{rd}
    \item \textbf{CentraleSupélec}: 36\textsuperscript{th}
    \item \textbf{École Polytechnique (X)}: 57\textsuperscript{th}
\end{itemize}

\vspace{0.5cm}

% --- SECTION 2: MILITARY ---
\section{2021-2022: Leadership in the Field}

Joining École Polytechnique means, first and foremost, becoming an officer. My journey began with total military immersion at the \textbf{Saint-Cyr Coëtquidan Military Academy}. Leaving the comfort of the classroom for the mud of the obstacle courses, I learned that intellectual capacity means little without the character to back it up.

Seeking to push my limits, I chose to perform my active service overseas, joining the \textbf{RSMA of Martinique} (Adapted Military Service Regiment).

\subsection*{A Lesson in Humility}
As an **Officer Cadet** and later **Platoon Leader**, I was entrusted with the command of 30 young recruits and the management of seasoned soldiers with over 20 years of experience.
At just 20 years old, I faced a stark reality: legitimacy is not granted by rank, but earned through exemplarity. Whether leading a march in the tropical heat or organizing logistics for a training camp, I learned that true leadership is about serving those you command. This operational experience, sharpened by commando training (CNEF), gave me a sense of responsibility that no textbook could provide.

% Placeholder for MILITARY PHOTO
\begin{figure}[h]
    \centering
    \framebox{\parbox{0.6\textwidth}{\centering \vspace{3cm} \textit{[Photo: In uniform at RSMA Martinique / Reviewing troops]} \vspace{3cm}}}
\end{figure}

\vspace{0.5cm}

% --- SECTION 3: POLYTECHNIQUE ---
\section{2022-2024: The Polytechnic Engineering Cycle}

Returning to the Palaiseau campus, I embraced the \textbf{multidisciplinary excellence} of the "X" curriculum. I refused to specialize too early, choosing instead to explore the full spectrum of science—from **Quantum Physics** and **Relativity** to **Macroeconomics** and **Fluid Mechanics**—while maintaining a rigorous core in Applied Mathematics. This breadth of knowledge has given me a unique ability to connect dots between disparate fields.

\subsection*{Bastille Day 2022: A Symbol of Service}
On July 14, 2022, I had the immense honor of marching down the Champs-Élysées under the flag of École Polytechnique. Parading in full \textit{Grand Uniforme} before the President of the Republic and the nation was more than a ceremonial duty; it was a profound moment of cohesion. Marching in perfect synchronization with my entire class represented the culmination of our military training and a tangible connection to the centuries of history we were inheriting.

\subsection*{Commitment to the Community}
Life at Polytechnique is deeply communal. I dedicated myself to shaping the student experience:
\begin{itemize}[label=\textbullet]
    \item \textbf{Bal de l'X:} As \textit{Dancer Manager}, I helped orchestrate one of France's most prestigious balls at the Opéra Garnier, managing artistic teams and logistics.
    \item \textbf{X-Forum:} As \textit{Web/Data Manager}, I bridged the gap between students and the professional world.
    \item \textbf{Sports:} Competition remained a pillar of my life. As a member of the **Basketball team**, winning the **Coupe de l'X** (inter-university tournament) was a highlight of team spirit and determination.
\end{itemize}

\subsection*{2023: Training the Next Generation}
In a rare opportunity, I returned to uniform in 2023 to supervise the Initial Military Training of the incoming X2023 cohort. Resuming my role as Platoon Leader—this time for my future peers—was a complex challenge. It required a subtle balance of authority and mentorship, reinforcing my belief that leadership is, above all, a human endeavor.

% Placeholder for CAMPUS PHOTO
\begin{figure}[h]
    \centering
    \framebox{\parbox{0.6\textwidth}{\centering \vspace{3cm} \textit{[Photo: July 14th Parade / Grand Uniforme / Sports Victory]} \vspace{3cm}}}
\end{figure}

\vspace{0.5cm}

% --- SECTION 4: MASTER ---
\section{2024-Present: The Pivot to Artificial Intelligence}

With a solid scientific foundation and a matured character, I chose to focus my energy on the defining technology of our time. I joined the highly selective \textbf{M2 IASD (AI, Systems, Data)} at Dauphine-PSL (co-accredited by ENS-Ulm).

This is where my journey converges. The rigor of my prep classes helps me navigate complex proofs in **Optimization** (Francis Bach, Clément Royer); my multidisciplinary background allows me to grasp the systemic implications of **Deep Reinforcement Learning** and **Computer Vision**. I am now fully dedicated to research, driven by the same curiosity that started this journey: understanding how the world works, and building tools to shape it.

\end{document}